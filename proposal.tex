\documentclass[12pt]{extarticle}
\usepackage[utf8]{inputenc}
\usepackage{cite}
\usepackage{url}

\title{Quantification of Uncertainty in Medical Image Segmentation}
\author{Hao Wang, 11812301}
\date{October 2020}

\begin{document}

\maketitle
\section*{Abstract}
\textbf{TODO}
\section{Introduction}
\paragraph{}
Recent development of deep learning algorithms has largely contributed to
the performance gain on natural image analysis, of which lots
of algorithms has been proposed to address its classification and semantic
segmentation problems and large-scale datasets with abundant annotations have been
carried out\cite{nair_precup_arnold_arbel_2020}.
However, due to
% inter-reader variations need more details and background
both lack of annotations and high inter-reader variations\cite{zhang2020disentangling}
in medical domain, the application of deep models into clinical diagnosis,
which is known to be sensitive to false-positive and true-negative cases that may leads
to severe consequences, is currently not feasible. 
For instance, the inter-annotator difference of glioblastoma segmentation
is pointed out to be in the range of 74-84\% \cite{6975210}.
Researches on inter-observer variance also found out
the large and wide-existing inconsistency among annotations of biomedical structures
\cite{Variability2019}\cite{interobserver2018}. Such high-variance dataset 
could result in poor performance of supervised machine learning algorithms,
which are known to be highly dependent on the quality of data as well as their annotations.
To overcome it, the most straight forward and widely-adopted method 
is to take the majority vote of experts' annotations, treating every 
experts' opinions equally significant\cite{6975210}, while STAPLE proposed by \cite{STAPLE} 
evaluates reliability of annotators and assigns weights to each of them when 
fusing annotations accordingly. Both of the approaches focuses on per image annotation fusion
while ignoring information across the whole dataset\cite{zhang2020disentangling}.
Lately, a challenge about Quantification of uncertainty 
in biomedical image segmentation has been carried
out on the purpose of seeking measures to quantify the uncertainty of 
medical image annotation. A discrete approximation of the uncertainty is also
proposed for the assessment of participants' results, which
takes in the averaged ground truth and the provided predicted probability map of a task
and computes and averages the dice score of the threshold of them with
discrete threshold values\cite{qubiq}. A high averaged dice score is then considered to 
be an indicating of good quantification of the inter-observer uncertainty.
Therefore, our goal is to firstly achieve a high score in the challenge. 
Then, since taking threshold averaged dice score treats certain and uncertain pixels equally 
and ignores the fact that most uncertain pixels lies around 
the contour of the lesion area, we are expecting to have a
better evaluation method for inter-reader variations. 
Finally, based on the metrics we derive, a further
optimized method could be explored and proposed.
\section{Related Works}
\paragraph{}
Uncertainty in deep learning can be inclusively divided into \textit{Aleatoric}
uncertainty and \textit{Epistemic} uncertainty \cite{kendall2017uncertainties}.
\textit{Aleatoric} uncertainty or data uncertainty typically occurs during data sampling and
observation error of observer and is irreducible even feeded with more data, which shares a
close connection with inter-observer variations. As for \textit{epistemic}
uncertainty, it refers to ambiguity of model prediction that can be reduce by feeding abundant data.
An example of it is that a model trained on only part of a dataset may over-confidently
predict unseen data into wrong classes. Recent development on bayesian neural network has made it possible
to quantify the \textit{epistemic} uncertainty by replacing scalar parameters with distributions such that 
the model outputs differs for each forward pass during test stage and that output variance reflects the 
certainty of the model. Such uncertainty could also be exploited to estimate \textit{aleatoric} uncertainty.
\cite{kohl2019probabilistic} proposed a method that involve uncertainties by
deploying variational auto-encoder(VAE) into U-net, the widely used model in medical image segmentation. 
The VAE here accounts for variations of annotations and represents an estimation of joint distribution that 
includes every pixels of the segmentation map. The framework is capable of providing multiple segmentation 
hypothesis with each estimating knowledge of an expert.
However, one of drawbacks of the bayesian approach is of the extra computation
and excessive training time. Some recent studies proposed to use a dropout network,
during the testing stage of which, the dropout is kept, predictions are sampled
using Monte Carlo methods and various uncertainties are computed based on the sampled results.
It is also reported that Effective performance can be achieved without any additional parameters even 
compared with non-bayesian methods. However, for such methods, the ambiguity obtained is
% Aleatoric & Epistemic uncertainty needs more explanation
not a direct representation of the \textit{Aleatoric} uncertainty,
the noise of data but \textit{epistemic} uncertainty, ambiguity in model
\cite{kendall2017uncertainties}. Therefore, the resulted uncertainty could not
be treated as the inter-reader variability but the reliability of the model itself.
\cite{nair2018exploring}. Study by Alex Kendall and Yarin Gal \cite{kendall2017uncertainties} 
combines both \textit{aleatoric} and \textit{epistemic} uncertainty in a unified framework 
\paragraph{}
Another study by \cite{zhang2020disentangling} used confusion matrices to
model each annotator. They hold the assumption that there is one and
only one ground truth annotation for an input of certain task settings and
that annotations are noised approximation of that
, which can be obtained by matrix multiplication of annotators' 
confusion matrices with ground truth. Then, by letting two convolution neural network backbone
where one outputs the estimated ground truth and the other accounts for the confusion matrices
of each annotator, and by forcing the matrix multiplication of them to be similar to the
real annotation, a unique ground truth could be learned by optimizing the cross entropy term
and another regularization term that minimize the trace of the confusion matrices.
Though the inter-observer variance not directly measured, the well-modeled annotators'
confusion matrices could be utilized to produce further results
as well as measurements of uncertainties.
Nevertheless, the correctness of the assumption of this approach
still remains to be verified and evaluated since there's no dataset available 
that fulfils their requirement by the publish time of their work.
\paragraph{}
\textbf{TODO: }

\section{Planned Method}
\paragraph{}
To model the inter-observer variability in the segmentation task, the most straight forward way is to 
directly estimate the annotations by each experts separately, with each channels of 
the probability map output representing one of them. However, since dice coefficient loss and cross entropy 
assign equal weights to every target area, the area with most disagreement, 
the area around the contour won't be handled carefully\cite{Kervadec_2021}.
The second thought is to follow the metric given by \cite{qubiq} and directly optimize on it by estimate threshold 
result of the averaged segmentation map at threshold value of $\frac{1}{M}, \frac{2}{M}, ..., \frac{M}{M}$, 
where $M$ is the number of annotators. A preliminary result is obtained and shown in section 4 using this method.
\paragraph{}
Another direction would be to follow the idea of \cite{zhang2020disentangling} and \cite{kohl2019probabilistic}, 
dividing the architecture into two components. 
While one accounts for generating estimated general segmentation map,
the other models annotators in a unified or separated manner. On combining the two components, a consistent segmentation
result based on each annotator could be obtained and various uncertainty metrics could be adopted for
the evaluation of the correctness of the uncertainty.
\section{Preliminary Results}
\section{Evaluation}
\textbf{TODO}

\section{Problem Formulation}
\begin{enumerate}
    \item Achieve high performance under QUBIQ evaluation metrics.
    \item Define a better metric for the assessment of uncertainties in medical image segmentation.
    \item Derive a method for the measurement of data uncertainties.
    \item Address uncertainties caused by inter-reader variations.
\end{enumerate}

\bibliographystyle{plain}
\bibliography{M335}

\end{document}
